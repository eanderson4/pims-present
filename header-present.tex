\documentclass[12pt]{beamer}

% For fast compiling
%\includeonlylecture{intro}

% -- PACKAGES -- 
\usepackage{pgf}
\usepackage{pgfplots}
\usepackage{tikz}
\usetikzlibrary{patterns}
\usepackage{amsmath,amssymb,amsopn}
\usepackage[latin1]{inputenc}
\usepackage[english]{babel}
\usepackage{booktabs,colortbl}
\usepackage{subfigure}
\usepackage{import}
%\usepackage{algorithm}
\usepackage{algorithm}
%\usepackage{algorithmic}
\usepackage[noend]{algpseudocode}

\usepackage{graphicx}
\usepackage{cleveref}
\graphicspath{{./img/}}
%\newtheorem{lemma}[theorem]{Lemma}




\usepackage{biblatex}
\addbibresource{ref.bib}
\addbibresource{jcc-bib.bib}
\addbibresource{msip-bib.bib}
\addbibresource{dfo-bib.bib}



%-- Beamer setup --%

% Colors
%\definecolor{wisconsin-gold}{rgb}{0.8,0.6,0}
\definecolor{lehigh-gold}{rgb}{1,0.9,0.1}
\definecolor{wisconsin-red}{rgb}{0.6,0,0}

\definecolor{french-blue}{rgb}{0,0.1254,0.62}
\definecolor{french-red}{rgb}{0.96,0.16,0.25}

 \definecolor{myblue1}{RGB}{35,119,189}
    \definecolor{myblue2}{RGB}{95,179,238}
    \definecolor{myblue3}{RGB}{129,168,207}
    \definecolor{myblue4}{RGB}{26,89,142}


\setbeamercolor*{structure}{fg=myblue1,bg=blue}
    \setbeamercolor*{palette primary}{use=structure,fg=white,bg=structure.fg}
    \setbeamercolor*{palette secondary}{use=structure,fg=white,bg=structure.fg!75!black}
    \setbeamercolor*{palette tertiary}{use=structure,fg=white,bg=structure.fg!50!black}
    \setbeamercolor*{palette quaternary}{fg=black,bg=white}

    \setbeamercolor*{item projected}{fg=red,bg=myblue3!80}
    \setbeamercolor*{block title example}{fg=white,bg=myblue4}
    \setbeamercolor*{frametitle}{fg=myblue4,bg=white}

    \setbeamertemplate{blocks}[rounded][shadow=true]

    \makeatletter
    \pgfdeclarehorizontalshading[frametitle.bg,frametitle right.bg]{beamer@frametitleshade}{\paperheight}{%
      color(0pt)=(myblue2);
      color(\paperwidth)=(white)}

    \defbeamertemplate*{footline}{mysplit theme}
    {%
      \leavevmode%
      \hbox{\begin{beamercolorbox}[wd=.5\paperwidth,ht=2.5ex,dp=1.125ex,leftskip=.3cm plus1fill,rightskip=.3cm]{author in head/foot}%
        \usebeamerfont{author in head/foot}\insertshortauthor
      \end{beamercolorbox}%
      \begin{beamercolorbox}[wd=.5\paperwidth,ht=2.5ex,dp=1.125ex,leftskip=.3cm,rightskip=.3cm plus1fil]{title in head/foot}%
        \usebeamerfont{title in head/foot}\insertshorttitle\hfill
        \insertframenumber/\inserttotalframenumber\hspace*{0.5em}
      \end{beamercolorbox}}%
      \vskip0pt%
    }
    \makeatother




\tikzset{
        hatch distance/.store in=\hatchdistance,
        hatch distance=10pt,
        hatch thickness/.store in=\hatchthickness,
        hatch thickness=2pt
    }

\makeatletter
    \pgfdeclarepatternformonly[\hatchdistance,\hatchthickness]{flexible hatch}
    {\pgfqpoint{0pt}{0pt}}
    {\pgfqpoint{\hatchdistance}{\hatchdistance}}
    {\pgfpoint{\hatchdistance-1pt}{\hatchdistance-1pt}}%
    {
        \pgfsetcolor{\tikz@pattern@color}
        \pgfsetlinewidth{\hatchthickness}
        \pgfpathmoveto{\pgfqpoint{0pt}{0pt}}
        \pgfpathlineto{\pgfqpoint{\hatchdistance}{\hatchdistance}}
        \pgfusepath{stroke}
    }


\makeatletter
    \pgfdeclarepatternformonly[\hatchdistance,\hatchthickness]{flexible hatch2}
    {\pgfqpoint{0pt}{0pt}}
    {\pgfqpoint{\hatchdistance}{\hatchdistance}}
    {\pgfpoint{\hatchdistance-1pt}{\hatchdistance-1pt}}%
    {
        \pgfsetcolor{orange}
        \pgfsetlinewidth{\hatchthickness}
        \pgfpathmoveto{\pgfqpoint{0pt}{0pt}}
        \pgfpathlineto{\pgfqpoint{\hatchdistance}{\hatchdistance}}
        \pgfusepath{stroke}
    }




% -- Images you want to use more than once --

%\pgfdeclareimage[width=1cm]{wisconsin-logo}{images/wcrest} 
%\logo{\pgfuseimage{wisconsin-logo}}


%\pgfdeclareimage[width=0.1\textwidth]{flag-small}{images/Flag_of_France}


% -- Jeff's MACROS -- %

\newcommand{\picframetitle}[2]{
    \vskip0.25em%
    \begin{centering}
      \begin{minipage}[c]{0.7\linewidth}
	\Large \alert{#1}
      \end{minipage} \hfill
      \begin{minipage}[c]{0.22\linewidth}
	\pgfimage[height=0.5in]{#2}
      \end{minipage}
      \par
    \end{centering}
}

\newcommand{\medpicframetitle}[2]{
    \vskip0.25em%
    \begin{centering}
      \begin{minipage}[c]{0.7\linewidth}
	\Large \alert{#1}
      \end{minipage} \hfill
      \begin{minipage}[c]{0.22\linewidth}
	\pgfimage[height=0.75in]{#2}
      \end{minipage}
      \par
    \end{centering}
}

\newcommand{\bigpicframetitle}[2]{
    \vskip0.25em%
    \begin{centering}
      \begin{minipage}[c]{0.7\linewidth}
	\Large \alert{#1}
      \end{minipage} \hfill
      \begin{minipage}[c]{0.22\linewidth}
	\pgfimage[height=1in]{#2}
      \end{minipage}
      \par
    \end{centering}
}

\newcommand{\bigpicframetitletwo}[2]{
    \vskip0.25em%
      \begin{minipage}[c]{0.6\linewidth}
	\Large \alert{#1}
      \end{minipage} \hfill
      \begin{minipage}[c]{0.32\linewidth}
	\hfill \pgfimage[height=0.8in]{#2}
      \end{minipage}
      \par
}

\newcommand{\bigpicframetitlethree}[2]{
    \vskip0.25em%
      \begin{minipage}[c]{0.6\linewidth}
	\Large \alert{#1}
      \end{minipage} \hfill
      \begin{minipage}[c]{0.32\linewidth}
	\hfill \pgfimage[height=0.8in]{#2}
      \end{minipage}
      \par
}

\newcommand{\bi}{\begin{itemize}}
\newcommand{\ei}{\end{itemize}}
\newcommand{\BI}{\begin{itemize}}
\newcommand{\EI}{\end{itemize}}
\newcommand{\BEN}{\begin{enumerate}}
\newcommand{\EEN}{\end{enumerate}}

\newcommand{\BBR}[1]{\begin{beamerboxesrounded}[shadow=true,width=0.96\textwidth]{\color{lehigh-gold}#1}}
\newcommand{\EBR}{\end{beamerboxesrounded}}

\newcommand{\BCS}{\begin{columns}}
\newcommand{\BC}[1]{\begin{column}{#1\linewidth}}
\newcommand{\EC}{\end{column}}
\newcommand{\ECS}{\end{columns}}

\newcommand{\MSS}{\medskip\alert{\hrule}\medskip}
\newcommand{\BSS}{\bigskip\alert{\hrule}\bigskip}
\newcommand{\st}{\; | \;}

\newcommand{\noprint}[1]{}
\newcommand{\exclude}[1]{}
\newcommand{\cB}{{\cal B}}
\newcommand{\cC}{{\cal C}}
\newcommand{\cD}{{\cal D}}
\newcommand{\cE}{{\cal E}}
\newcommand{\cF}{{\cal F}}
\newcommand{\cG}{{\cal G}}
\newcommand{\cI}{{\cal I}}
\newcommand{\cK}{{\cal K}}
\newcommand{\cL}{{\cal L}}
\newcommand{\cN}{{\cal N}}
\newcommand{\cO}{{\cal O}}
\newcommand{\cP}{{\cal P}}
\newcommand{\cQ}{{\cal Q}}
\newcommand{\cR}{{\cal R}}
\newcommand{\cS}{{\cal S}}
\newcommand{\cT}{{\cal T}}
\newcommand{\cU}{{\cal U}}
\newcommand{\cX}{{\cal X}}
\newcommand{\cV}{{\cal V}}
\newcommand{\cZ}{{\cal Z}}

\newcommand{\magE}{\left| \mathcal{E} \right|}

\newcommand{\defeq}{\stackrel{\rm def}{=}}
\newcommand{\Expect}{\mathbb{E}}

\newcommand{\orb}{\operatorname{orb}}


\newcommand{\ry}{\pmb{y}}
\newcommand{\rye}{\pmb{y_e}}

\makeatletter
\def\BState{\State\hskip-\ALG@thistlm}
\makeatother


\defbeamertemplate*{title page}{customized}[1][]
{
  \usebeamerfont{title}\inserttitle\par
  \usebeamerfont{subtitle}\usebeamercolor[fg]{subtitle}\insertsubtitle\par
  \bigskip
  \bigskip
  \bigskip
  \begin{tabular}{ c l}
    \begin{tabular}{c}
  \usebeamercolor[fg]{titlegraphic}\inserttitlegraphic \\

  \usebeamerfont{institute}\insertinstitute
    \end{tabular} 
    & 
    \begin{tabular}{l}
    \usebeamerfont{author}\insertauthor \\
    \usebeamerfont{date}\insertdate
    \end{tabular} 
\end{tabular}
}
